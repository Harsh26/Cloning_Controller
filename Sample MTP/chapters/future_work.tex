\chapter{Future Work}

\bigbreak
Every device has its environment set up in a network. The environment can vary with different operating systems. So, having a platform that works in a network of systems with multiple environments is a game-changer. The current version of the TarPy has some constraints on working in a cross-platform environment. The agent can move from Linux to Windows environment; however, the movement of agents from Windows to Linux environment is not supported.\par

\bigbreak
The current version of TarPy is not tested with embedded OS such as Raspbian OS. Compatibility with such OS can give agents access to different sensors, actuators, GPIO, etc., connected with the embedded system. The agent can use this data to learn about its environment. Here, TarPy can then allow them to develop deep learning models on such data in a federated manner.\par


\bigbreak
The platform is tested on a couple of Algorithms like traversing in a Circular fashion, Distributed Minimum Spanning Tree Algorithm, Maekawa's Algorithm for mutual exclusion in a distributed system. There can be use cases where TarPy will need some improvement at the Prolog level. In that case, it will require some basic knowledge of Prolog to enhance the features of TarPy.\par
